
\chapter{PROPUESTA DE TESIS}
%%The prevalence of electronic media in the lives of young
%%children means that they are spending an increasing number
%%of hours per week in front of and engaged with screens of all
%%kinds, including televisions, computers, smartphones, tablets,
%%handheld game devices, and game consoles

%%With guidance, these various
%%technology tools can be harnessed for learning and development;
%%without guidance, usage can be inappropriate and/or
%%interfere with learning and development.

%% A DAY IN THE LIFE OF A ECE TEACHER
% The issues were identified as “insufficient time”, a “lack of resources” and also “working with
% parents” and these are discussed below

\section{Planteamiento del problema}

%% Alfabetizacion inicial
% un aprendizaje lector insuficiente durante los primeros años de la educación básica 
% limita los niveles de procesamiento cognitivo y el acceso al conocimiento, 
% en las diversas áreas de contenido de los programas escolares.

% recientemente se ha aceptado que muchos de los procesos más importantes para el 
% desarrollo de la alfabetización tienen lugar antes de que el alumno entre por
% primera vez en un aula (Pressley, 1999). 

El estado de desarrollo de las sociedades y los individuos que la componen est'an fuertemente
asociado. Una sociedad con un alto nivel de pobreza es indicativa de un bajo acceso a la educaci'on, 
y se vincula con la reproducci'on del analfabetismo. Seg'un \citet{infante13} "no hay posibilidad
e alcanzar una democracia efectiva, mientras gran parte de la poblaci'on se mantenga fuera del
acceso a la lengua escrita". La alfabetizaci'on es entonces un elemento fundamental para el debido funcionamiento de la sociedad, ha de ser vista como un derecho universal, y es deber de la sociedad misma promoverla. Sin embargo, se debe entender la alfabetizaci'on no como un proceso
simple que s'olo comprende leer y escribir, sino como a la puerta de entrada al aprendizaje
permanente. 

{\color{red} Necesito un p'arrafo de puente. Debe hablar sobre la educaci'on inicial,
el aprendizaje de la lectura (metodos mixto), y como pueden las TICs ayudar en esto.}

El r'apido avance de las Tecnologías de la Información y la Comunicación (TIC) influye
significativamente sobre diferentes 'ambitos sociales, como en la econom'ia, la cultura, y 
la educaci'on. De acuerdo con \cite{CurriculoSEPB} los egresados y las egresadas del subsistema
de Educaci'on Primaria Bolivariana debe poseer entre sus caracter'iasticas el uso
de las TIC en el proceso de aprendizaje, as'i mismo 'estas forman parte de los ejes
integradores del subsistema, lo que significa que todos los procesos educativos del subsistema
deben considerar las TIC para orientar las experiencias del aprendizaje, as'i como fomentar valores,
actitudes y virtudes.

Para la incorporaci'on de las TIC en la educaci'on se debe tomar en cuenta, entre otras cosas,
la cantidad, calidad y costos de software desarrollado espec'ificamente con fines educativos, y
enfoncados a ni'nos y ni'nas de corta edad, o c'omo el uso de las computadoras ha sido asociado 
comunmente con el entretenimiento y no con actividades que desarrollen las destrezas b'asicas 
necesarias para un desenvolvimiento 'optimo \cite{Archila}.

Favorecer el uso de las TIC en la educaci'on depende entonces de la accesibilidad que se le de
a los miembros de la comunidad a las herramientas de hardware (computadoras, tablets, tel'efonos
m'oviles), y software (programas, aplicaciones, videojuegos). Lo primero es atacado por iniciativas
como el Proyecto Canaima Educativo {\color{red} (incluir referencia)}. Lo segundo implica el desarrollo de herramientas bien fundamentadas en las diferentes metodolog'ias del aprendizaje,
 basadas en licencias libres, y con un alto nivel de portabilidad, mantenibilidad y extensibilidad.
 
El cambio curricular que incluye a las TIC como eje integrador del subsistema de Educaci'on
Primaria Bolivariana data del a~no 2007, por lo que gran parte del software que se utiliza
es el que ya exist'ia para entonces. Esto significa que no hay software actualizado, acorde con 
las nuevas tecnolog'ias, algo fundamental un entorno que evoluciona tan velozmente como lo son
las TIC. As'i mismo hay que tomar en cuenta que en la actualidad los ni'nos y ni'nas nacen, y se
desarrollan de la mano de la tecnolog'ia, est'an fuertemente adaptados a esta, y siguen 
f'acilmente los patrones y tendencias actuales, llegando incluso a rechazar aquello que consideren
desfasado.

{\color{red} Hace falta un p'arrafo de cierre.}

\section{Objetivo General}

Desarrollar una herramienta que permita {\color{red}emplear} el m'etodo mixto de ense'nanza de la lectura, enfocada a ni'nos y ni'nas durante la etapa de educaci'on inicial, que siga las tendencias
tecnol'ogicas actuales, y sea altamente portable, mantenible y extensible. 


\section{Objetivos Espec'ificos}


\section{Justificaci'on de la propuesta}



