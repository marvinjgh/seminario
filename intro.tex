%% Los cap'itulos inician con \chapter{T'itulo}, estos aparecen numerados y
%% se incluyen en el 'indice general.
%%
%% Recuerda que aqu'i ya puedes escribir acentos como: 'a, 'e, 'i, etc.
%% La letra n con tilde es: 'n.

%Existen dos tipos de citas bibliograf'icas: usa \verb|\citep{..}| para
%citas en \emph{par'entesis} y \verb|\citet{..}| para citas
%en el \emph{texto}. Por ejemplo, estudios reciente han mostrado nuevos e
%interesantes modelos que se pueden aplicar para reformular teor'ias
%f'isicas~\citep{NewCam97}. Mientras que, el trabajo de \citet{Rofl06} fue
%considerado muy divertido por una significativa fracci'on de la comunidad
%de investigadores. Tambi'en es posible citar a varios trabajos en una sola
%referencia \citep{Lamport86,Knuth84}.
%
%Estos comandos para producir citas bibliograficas son provistos por
%el paquete \textsf{natbib}. Para obtener m'as informaci'on, consulta la
%documentaci'on de ese paquete~\citep{doc:natbib}. Por su parte, en
%la documentaci\'on de \textsf{geometry} puedes encontrar detalles
%adicionales sobre el sistema para ajustar los m'argenes del
%documento~\citep{doc:geometry}. Lo que sigue
%es un mont'on de texto sin sentido en lat'in que utilizaremos para llenar
%algunas p'aginas.



\prefacesection{Introducci'on}


%Learning to read is an important educational goal. For both
%children and adults, the ability to read opens up new worlds
%and opportunities. It enables us to gain new knowledge, enjoy
%literature, and do everyday things that are part and parcel of
%modern life

%a nivel universal, se consideran tres
%aprendizajes esenciales para la vida: la lectura, la escritura y el pensamiento lógico-matemático.

%% Reading development involves the participation of children, parents,
%% educators and the community as a whole. 

%% A child’s initial contact with words and symbols happens before going to
%% school.
