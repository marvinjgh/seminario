
\chapter{MARCO TE'ORICO}

%% --------------------- RESUMEN DEL CAPITULO ---------------------


\section{Alfebitzaci'on}

%%ALFABETIZACION Y EDUCACION - UNESCO \infante13

% --Alfabetizacion funcional:  “está alfabetizada toda persona que puede leer y escribir, comprendiéndola,
% una breve y sencilla exposición de hechos relativos a su vida cotidiana”.
% “El objetivo de la alfabetización funcional estaba determinado por la urgencia de
% movilizar, formar y educar la mano de obra aún subutilizada, para volverla más
% productiva, más útil a ella misma y a la sociedad” 

% Las ideas progresistas se incorporan al Simposio Internacional de Alfabetización
% celebrado en Persépolis 1975. En la declaración final, se define la “alfabetización
% no solo como el aprendizaje de la lectura, la escritura y el cálculo, sino como una
% contribución a la liberación de la persona y a su pleno desarrollo. Así concebida, la
% alfabetización crea las condiciones para la adquisición de una conciencia crítica de
% las contradicciones y los objetivos de la sociedad en que se vive; también estimula
% su iniciativa y participación en la creación de un proyecto capaz de actuar en el
% mundo, de transformarlo, y de definir los objetivos de un auténtico desarrollo
% humano”(IIALM, 1977, p. 636).

% -- La alfabetización entendida como un continuum de habilidades
%  “la habilidad de entender y utilizar información impresa en actividades cotidianas en el hogar,
% la comunidad y el trabajo”.

% -Hacia el concepto de alfabetización como puerta de entrada al aprendizaje permanente
% La alfabetización comienza a ser vista como un cimiento fundamental de todos los aprendizajes.

El concepto de alfabetizaci'on ha ido evolucionando a trav'es del tiempo. La estructura de la
sociedad est'a asociada historicacmente con la alfabetizaci'on, por tanto, los cambios culturales,
pol'iticos, y econ'omicos que sufre la sociedad afectan directamente la definici'on de alfabeticaci'on.

Hacia los a'nos 50 se ten'ia el concepto tradicional de alfabetizaci'on, y se consideraba alfabeta 
a aquellas personas que supieran decodificar los signos necesarios para leer y escribir \cite{infante13}, sin tomar en cuenta el uso que se le pudiera dar a estas habilidades.

\subsection{Alfabetizaci'on funcional}
En la d'ecada de los 60 se introduce el concepto de ``\textbf{alfabetizaci'on funcional}" que se definida por \citet{Unesco1970} como ``cualquier operaci'on de alfabetizaci'on concebida como un componente de los proyectos de desarrollo econom'omico y social". Se separa de la alfabetizaci'on tradicional al enlazar el aprendizaje de la lectura y la escritura con la formaci'on de los individuos como profesionales, y entiende que estas habilidades permiten la adquisici'on de aptitudes profesionales y conocimientos empleables en 'ambitos concretos. Sin embargo este enfoque inicial estaba orientado a formar y educar la mano de obra con la intenci'on de hacerla m'as productiva. M'as adelante las ideas progresistas comienzan a vincular la alfabetizaci'on con la concientizaci'on y el cambio social. A finales de la d'ecada de los 70 se incorpora una nueva definici'on de alfabetizaci'on funcional: ``Es alfabeto funcional la persona que puede emprender aquellas actividades en que la alfabetizaci'on es necesaria para la actuaci'on eficaz en su grupo y comunidad y que le permitan asimismo seguir valie'endose de la lectura, la escritura y la aritm'etica al servicio de su propio desarrollo y del desarrollo de la comunidad". {\color{red} Citar Acta UNESCO 1978}

\subsection{La alfabetizaci'on como puerta de entrada al aprendizaje permanente}
En la Declaraci'on Universal de los Derechos Humanos a final de los a'nos 40 se establece que ``toda persona tiene derecho a la educaci'on". Sin embargo, cuarena a'nos despu'es el panor'ama mundial en lo que acceso a la educaci'on se refiere dejaba mucho desear. La UNESCO convoca la Conferencia Mundial sobre Educaci'on para Todos, que se reune en Jomtien, Tailandia, en el a'no 1990. Y establece como su objetivo principal la \textbf{satisfacci'on de las necesidades b'asicas del aprendizaje}. Entre ellas se encuentran herramientas escenci'ales como la lectura y la escritura, la expresi'on oral, el c'alculo, y tambien contenidos b'asicos de aprendizaje necesarios para que los seres humanos puedan desarrollar plenamente sus capacidades, vivir y trabajar con dignidad, tomar decisiones fundamentales y \textbf{continuar aprendiendo}. Ya no se define la educaci'on b'asica como un fin, sino como una base para el aprendizaje y desarrollo humano permanetes, que permita a los pueblos construir segura nuevos niveles y tipos de educaci'on \cite{Unesco1990}

Se entiende entonces que el aprendizaje transcurre de forma continua a lo largo de la vida, y que la alfabetizaci'on es una puerta de entrada al aprendizaje. La alfabetizaci'on es vista como una base para todos los aprendizajes. El proceso de alfabetizaci'on debe brindarle a las personas herramientas que le permitan desenvolverse en su medio cotidiano y seguir aprendiendo \cite{infante13}. Sin embargo, cada pa'is tiene sus propias necesidades b'asicas de aprendizaje y formas de satisfacerlas, inherente a su cultura e identidad hist'orica, y que adem'as estas cambian con el tiempo.


\section{La Educaci'on Inicial}

%% Venezuela AEPI
% La atención educativa e integral a la primera infancia, se define como Educación
% Inicial, la cual se concibe como el proceso de desarrollo y aprendizaje de los niños
% niñas, como un continuo que se inicia desde la gestación hasta los 6 años de edad o
% cuando se incorpore a la educación primaria y tiene como finalidad propiciar el
% máximo desarrollo de las potencialidades socio-afectivas, cognitivas, lingüísticas,
% motoras y físicas; considerando sus experiencias socio-educativas, intereses y
% necesidades.

Si la alfabetizaci'on es una puerta de entrada al aprendizaje, la educaci'on inicial es la puerta de entrada a la alfabetizaci'on. En venezuela, previo al a'no 1999, se reconoc'ia la educaci'on obligator'ia para ni'nos y ni'nas a partir del nivel preescoler, el cual incia los 4 a'nos de edad. Luego de la reforma constitucional de ese a'no se establece que ``la educaci'on es obligatoria en todos sus niveles, desde el maternal hasta el nivel medio diversificado" \cite{UnescoAEPI}

En el a'no 2007 nace el subsitema de \textbf{Educaci'on Inicial Bolivariana}, el cual es definido en \cite{CurriculoSEIB} como el subsistema que ``brinda atenci'on educativa al ni'no y la ni'na entre cero (0) y seis (6) a'nos de edad, o hasta su ingreso al subsistema siguiente, concibi'endolo como sujeto de derecho y ser social integrante de una familia y de la comunidad, que posee caracter'isticas personales, sociales, culturales y lugu'isticas propias y que aprende en un proceso constructivo e integrado en lo afectivo, lo l'udico y la inteligencia, a fin de garantizar su desarrollo integral."

El subsistema de Educaci'on Inicial se divide en dos niveles, el nivel maternal que comprende a ni'nos y nin'as de cero (0) a tres (3) an'os, y el nivel preescolar el cual incluye a ni'nos y ni'nas entre cuatro (4) y seis (6) a'nos. Viene al caso repasar algunos de los objetivos del nivel preescolar:

\begin{itemize}
	\item Fortalecerlas potencialidades, habilidades y destrezas de los ni'nos y las ni'nas, a fin de que 'estos y 'estas puedan ingresar con 'exito a la \textbf{Educaci'on Primaria Bolivariana}.
	\item Fomentas las diferentes formas de comunicaci'on (ling\"{u}'istica, gestual y escrita).
	\item Potenciar el desarrollo de la percepci'on, la memoria, la atenci'on y la inteligencia a trav'es de la afectividad y las diversas actividades l'udicas adecuadas a la edad.
	\item Promover la manipulaci'on, descubrimiento y conocimiento de alginos recursos tecnol'ogicos de su entorno.
\end{itemize}

{\color{red} P'arrafo de cierre.}

\section{La Educaci'on Primaria}
Continuando bajo el marco del \textbf{Sistena Educativo Bolivariano}, los ni'nos y las ni'nas desde los seis (6) hasta los (12) a'nos de edad deben integrarse en el subsitema de \textbf{Educaci'on Primaria Bolivariana (EPB)}. En \cite{CurriculoSEPB} se describe la finalidad de este susbistema como:

\begin{quote}
	...formar ni'nos y nin;as activos, reflexivos, cr'itios e independientes, con elevado inter'es por la actividad cient'ifica, humanista y art'istica; con un desarrollo de la compresi'on, confrontaci'on y verificaci'on de su realidad por s'i mismos y s'i mismas; con una conciencia que les permita aprender desde el entorno y ser cada vez m'as participativosm protag'onivos y corresponsables en su actuaci'on en la escuela, familia y comunidad. Asimismo, promover'a actitudes para el amor y el respeto hacia la Patria, con una visi'on de unidad, integraci'on y cooperaci'on hacia los pueblos latinoamericanos, caribe'nos y del mundo.
\end{quote}

El susbistema de \textbf{EPB} hace incapi'e en impulsar el dominio de las \textbf{Tecnolog'ias de la Informacion y Comunicaci'on}, las cataloga como uno de sus \textbf{ejes integradores} es decir, deben ser consideradas en todos los procesos educativos del subsistema como una herramienta para la organizaci'on e integraci'on de los saberes y orientaci'on de las experiencias de aprendizaje. Asimismo define cuatro 'areas de aprendizaje: 

\begin{itemize}
	\item Lenguaje, comunicaci'on y cultura.
	\item Matem'aticas, ciencias naturales y sociedad.
	\item Ciencias sociales, ciudadan'ia e identidad.
	\item Educaci'on f'isica, deportes y recreaci'on.
\end{itemize}

Seis (6) a'nos es la duraci'on del subsistema de \textbf{EPB}, divididos de forma anual. Cada a'no plantea cumplir con ciertos objetivos en cada una de las 'areas de aprendizaje con el fin de inculcar ciertas caracter'isticas en los egresados y egresadas del subsistema, tales como:

\begin{itemize}
	\item Su identidad venezolana, con una conciencia y visión latinoamericana,
	caribeña y universal.
	\item El dominio pr'actico de un idioma materno (castellano, ind'igena y otros), al escuchar, leer y construir (oralmente y por escrito) diferentes tipos de textos de forma clara, emotiva, coherente, fluida y correcta, sobre la base de sus experiencias personales.
	\item Las Tecnologías de la Informaci'on y la Comunicaci'on (TIC) en el proceson de aprendizaje.
	\item El pensamiento para organizar y transformar la informaci'on recibida,
	elaborando nuevos conocimientos. 
\end{itemize}

Cumplir con ese cometido requiere de un enfoque sistem'atico, de modo que el susbsitema de \textbf{EPB} trata a'no a an'no cumplir con ciertos objetivos en pro de su fin. Estos objetivos se definen en cada 'area de aprendizaje, y es pertinente para esta investigaci'on repasar algunos de los objetivos del area de aprendiza \textbf{Lenguaje, comunicaci'on y cultura}:

\begin{itemize}
	\item Decodifiacic'on de simbolos para darle significado como palabras.
	\item Seguimiento de instrucciones orales al realizar diversas actividades.
	\item Decodificaci'on de s'imbolos y c'odigos ling\"{u}'isticos.
	\item Diferenciaci'on de la escritura: grafismo, escritura sin control de cantidad, escritura fija, correspondencia entre el sonido y lo que escribe.
	\item Establecimiento de correspondencia entre el fonema y el grafema con valor sonoro convencional.
	\item Descubrimiento del sist'ema alfab'etico convencional: g'enero y n'umero, uso de las may'usculas y del punto.
	\item Identificaci'on, ejercitaci'on y separaci'on de las palabras en s'ilaba y clasificaci'on de las palabras seg'un el n'umero de s'ilabas.
\end{itemize}

\section{Influencia de las TIC en la educaci'on}


\section{Juegos serios}




