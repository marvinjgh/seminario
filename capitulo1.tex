
\chapter{MARCO TE'ORICO}

%% --------------------- RESUMEN DEL CAPITULO ---------------------


\section{Alfebitzaci'on}

%%ALFABETIZACION Y EDUCACION - UNESCO \infante13

% --Alfabetizacion funcional:  “está alfabetizada toda persona que puede leer y escribir, comprendiéndola,
% una breve y sencilla exposición de hechos relativos a su vida cotidiana”.
% “El objetivo de la alfabetización funcional estaba determinado por la urgencia de
% movilizar, formar y educar la mano de obra aún subutilizada, para volverla más
% productiva, más útil a ella misma y a la sociedad” 

% Las ideas progresistas se incorporan al Simposio Internacional de Alfabetización
% celebrado en Persépolis 1975. En la declaración final, se define la “alfabetización
% no solo como el aprendizaje de la lectura, la escritura y el cálculo, sino como una
% contribución a la liberación de la persona y a su pleno desarrollo. Así concebida, la
% alfabetización crea las condiciones para la adquisición de una conciencia crítica de
% las contradicciones y los objetivos de la sociedad en que se vive; también estimula
% su iniciativa y participación en la creación de un proyecto capaz de actuar en el
% mundo, de transformarlo, y de definir los objetivos de un auténtico desarrollo
% humano”(IIALM, 1977, p. 636).

% -- La alfabetización entendida como un continuum de habilidades
%  “la habilidad de entender y utilizar información impresa en actividades cotidianas en el hogar,
% la comunidad y el trabajo”.

% -Hacia el concepto de alfabetización como puerta de entrada al aprendizaje permanente
% La alfabetización comienza a ser vista como un cimiento fundamental de todos los aprendizajes.


\section{La educaci'on inicial}

%% Venezuela AEPI
% La atención educativa e integral a la primera infancia, se define como Educación
% Inicial, la cual se concibe como el proceso de desarrollo y aprendizaje de los niños
% niñas, como un continuo que se inicia desde la gestación hasta los 6 años de edad o
% cuando se incorpore a la educación primaria y tiene como finalidad propiciar el
% máximo desarrollo de las potencialidades socio-afectivas, cognitivas, lingüísticas,
% motoras y físicas; considerando sus experiencias socio-educativas, intereses y
% necesidades.

\section{Influencia de las TIC en la educaci'on inicial}


\section{Juegos serios}




